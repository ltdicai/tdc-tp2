
\documentclass[a4paper,spanish]{article}
\usepackage[a4paper,margin=1.5cm,top=0.5cm,bottom=0.5cm]{geometry}
\usepackage[pdftex]{graphicx}
\usepackage{subfigure} % subfiguras

% Paquetes varios, entre otras cosas, para poder escribir con acentos!

\usepackage[T1]{fontenc}
\usepackage{anysize} 
\usepackage[utf8]{inputenc}
\usepackage[spanish]{babel}
\usepackage{hyperref}
\usepackage[pdftex]{graphicx}
\usepackage{amsfonts}
\usepackage{amsmath}
\usepackage{amssymb}
\usepackage{array} 
\usepackage{tabularx}
\usepackage{textcomp}
\usepackage{pdfpages}
\usepackage[small]{caption}

%\usepackage{newclude}

\usepackage{fancyhdr}

\usepackage{paquetes/caratula}		% Si no tienen este paquete en sus paquetes de latex, 

%\hoffset=-2.5cm
%\textwidth=17cm
%\parskip=1ex

\pagestyle{fancy}

% Paquete para poder escribir los simbolos de naturales, enteros, reales...
% Si no lo tienen, hagan --> sudo apt-get install texlive-fonts-extra
% La cagada es que pesa unos 100 MB...

% \usepackage{dsfont}

% Paquetes para escribir algoritmos:

\usepackage{paquetes/algorithm}		% Si no tienen este paquete en sus paquetes de latex, 
                                  % este archivo tiene que estar en la carpeta donde estan nuestros .tex
\usepackage{paquetes/algorithmic}	% Idem que el de arriba

%\input{spanishAlgorithmic}	% Este SI o SI tiene que estar en la carpeta donde estan nuestros  .tex

%\usepackage{moreverb}
%\usepackage{verbatim} %entorno verbatim para codigo fuente

% Aca pueden ver bien como se usa el paquete...
% http://www.rosapolis.net/2008/04/21/escribir-algoritmos-en-latex/

\DeclareGraphicsExtensions{.bmp,.png,.pdf,.jpg,.eps}
\hypersetup {colorlinks=false, pdfborder={0 0 0}}
\newcommand{\fullref}[1]{sección \ref{#1}}
\newcommand{\quotel}{\textquotedblleft}
\newcommand{\quoter}{\textquotedblright \ }

% Cosas del enunciado de ellos

\parskip = 11pt
\newcommand{\real}{\hbox{\bf R}}

% No deja sangria al comienzo de los parrafos!
%\setlength\parindent{0pt}

\begin{document}

\materia{Teoría de las Comunicaciones}
\submateria{}
\titulo{TP2: Rutas en Internet}
\subtitulo{}
%\grupo{Nombre Grupo}
\integrante{Benitti, Raul}{592/08}{raulbenitti@gmail.com}
\integrante{Castro, Damian}{326/11}{ltdicai@gmail.com}
\integrante{Lizana, Helen}{118/08}{hsle.22@gmail.com}
\integrante{Grenier, Michelle}{418/10}{michelle.grenier@hotmail.com}



\fecha{08 de junio de 2016}

\maketitle

\tableofcontents
\newpage

%\input{abstract.tex}
%\newpage

%\input{jerarquia.tex}
%\newpage

\section{Introducción}

Internet es posible gracias a un conjunto de miles de redes interconectadas entre sí.
La conexion entre redes de distintos continentes se realiza por medio de cables submarinos capaces de transportar grandes volumenes de datos por segundo.
En este trabajo experimentaremos con herramientas y técnicas frecuentemente utilizadas para el análisis de redes.
Analizaremos que tan factible resulta utilizar los datos estadisticos conseguidos mediante una implementacion propia de un traceroute, para detectar saltos intercontinentales en las rutas que atraviesan los paquetes en Internet.  
Se nos pide utilizar como objetivos cuatro universidades ubicadas en diferentes partes del mundo, y a distancias variadas.

%%Los conceptos teóricos sobre los que basaremos el analisis se presentan a continuación: 
\section{Marco teórico}

\subsection{Protocolo de Mensajes de Control de Internet- ICMP}

Forma parte del conjunto de protocolos IP (RFC 792). Los mensajes ICMP son comúnmente generados en respuesta a errores en los datagramas de IP para diagnóstico y ruteo. Estos mensajes son construidos en el nivel de capa de red y se encuentran dentro de los paquetes de IP estándar.

\subsection{Traceroute}

Es una herramienta de diagnóstico que permite seguir la pista de los paquetes que vienen de un host. Se obtiene además una estadistica del RTT o latencia de red de eso paquetes.
Exiten varias maneras de implementar esta herramienta, usualmente se envian paquetes IP donde se incrementa progresivamente el campo Time To Live (TTL), este campo sirve para que un paquete no permanezca en la red de forma indefinida. Cuando un nodo de la red elimina un paquete, envía al emisor un mensaje ICMP indicando una incidencia. Traceroute usa esta respuesta para averiguar la dirección IP del nodo que desechó el paquete, y asi sucesivamente hasta que el paquete llegue a su destino.

\subsection {Round Trip Time}

El RTT es el tiempo que tarda un paquete en ir y volver desde un nodo A (el origen) a un nodo B (el destino) dentro de una red. 
Cuando se trata de enlaces punto a punto, se define como 2 * Delay. 
Si bien a nivel de enlace puede realizarse una estimación relativamente confiable del Delay a partir de variables conocidas (ancho de banda, velocidad de propagación del medio, etc), a nivel de red el RTT de un paquete IP queda sujeto a la ruta que este toma.
Así, el RTT de un paquete que viaja entre varias redes interconectadas depende de variables desconocidadas de los enlace intermedios, y empiezan a cobrar mayor importancia factores como la congestión de los routers intermedios. Por lo tanto, en este trabajo solo consideraremos al RTT en su sentido empírico
	

\newpage

\section{Experimentos}

Como mencionamos anteriormente, el análisis de paquetes de una red puede utilizarse para inferir información sobre la actividad y topología de la red. En este trabajo aprovecharemos esta capacidad para dilucidarqué protocolos se distinguen del resto, cuál es la incidencia de los paquetes ARP y cuáles son los nodos destacados de las redes.
Realizamos cuatro experimentos para obtener datos, uno sobre cada una de las siguientes redes:
\begin{itemize}
	\item Red1: Red wiFi de un laboratorio del DC
	\item Red2: Red Wifi de un bar Starbucks
	\item Red3: Red Ethernet en un ámbito laboral
	\item Red4: Red Ethernet en un ámbito laboral
\end{itemize}


Modelamos estas redes como dos fuentes de información distintas:
\begin{enumerate}
	\item $S$: este modelo fue dado por la cátedra. El alfabeto se define como los protocolos enviados dentro de los paquetes Ethernet capturados durante el experimento. Así mismo, consideramos como función de probabilidad a la frecuencia de cada símbolo dentro del experimento, donde marcamos como ocurrencia de un evento a la observación de un protocolo al capturar un paquete.
	\item $S1$: con este modelo deseamos poder distinguir los nodos relevantes de una red dada. Para ello, definimos el alfabeto de $S1$ como las direcciones MAC de los paquetes del protocolo ARP. En este caso también tomamos como función de probabilidad a la frecuencia de cada dirección MAC dentro del total observado, pero contabilizando las ocurrencias de cada MAC segun se muestra en el Cuadro \ref{ARP}.
La decisión de utilizar direcciones MAC en lugar de direcciones IP radica en el hecho de querer identificar los nodos físicos dentro de la red observada. Utilizar directamente direcciones IP podría llevar a malinterpretar la topología de la red: por ejemplo, podríamos considerar como relevantes a varios host con distintas IP que se encuentran fuera del sistema en estudio, y no notar que todo el táfico debe pasar por un único nodo propio. Por otra parte, decidimos utilizar las direcciones MAC tal como se muestra en el Cuadro \ref{ARP} pués consideramos que brindan la mayor información acerca de la actividad de cada nodo en la red: enviar un paquete ARP nos da información sobre la existencia del host, y de recibir un paquete ARP (es decir, la existencia de un paquete IS\_AT destinado a un nodo particular) podemos deducir que el nodo destino seguramente continue con envío de más paquetes (es decir, tenga actividad inmediata).

	
		\begin{table}
			\centering
		\begin{tabular}{l l}
				Observación & Eventos contabilizados \\
				\hline
				Paquete WHO\_HAS, MAC origen & 1 evento \\
				Paquete WHO\_HAS, MAC destino & 0 evento \\
				Paquete IS\_AT, MAC origen & 1 evento\\
				Paquete IS\_AT, MAC destino & 1 evento\\
		\end{tabular}
		\caption{Contabilización de eventos para $S1$}
		\label{ARP} 
		\end{table}
\end{enumerate}

\subsection{Herramientas de sniffing}

Para capturar y procesar la información, utilizamos tanto el programa \textit{Wireshark} como dos script (capturar.py, identificar.py), escritos en Python, utilizando la librería para análisis de redes \textit{scapy}. Ambas herramientas hacen uso del modo promíscuo de la placa de red, en el cual se capturan no solo los paquetes dirigidos a el host que esta capturando, sino todos los paquetes que se envíen por el medio.

\subsubsection{Implementación de $S$: capturar.py}
En su forma de ejecución básica, el script muestra por pantalla cada paquete que captura hasta que sea detenido con una interrupción (CTRl+C). Al finalizar, se muestra 
\begin{enumerate}
	\item el total de paquetes capturados
	\item los protocolos observados (junto con la cantidad de veces que se observo cada uno)
	\item la entropía correspondiente modelo $S$. 
\end{enumerate}
Si bien incorporamos varias opciones de ejecución (ejecutar el comando con la opción \textit{-h}.), la forma más sencilla corresponde a 
\begin{verbatim}
sudo python capturar.py -i <interfaz_de_captura>
\end{verbatim}

\subsubsection{Implementación de $S1$: identificar.py}
Este script es similar a capturar.py, pero en lugar de analizar los protocolos de cada paquete, filtra solo los paquetes ARP e implementa el modelo de fuente $S1$. Al igual que capturar.py, el script muestra por pantalla cada paquete que captura hasta que sea detenido con una interrupción (CTRl+C). Al finalizar, devuelve
\begin{enumerate}
	\item un diccionario donde se mapea direcciones MAC con direcciones IP
	\item las direcciones MAC observadas (junto con la cantidad de veces que se observo cada una)
	\item la entropía correspondiente modelo $S1$. 
\end{enumerate}
Para ver opciones de ejecución, ejecutar el comando con \textit{-h}.). la forma más sencilla corresponde a 
\begin{verbatim}
sudo python identificar.py -i <interfaz_de_captura>
\end{verbatim}


\subsection{Análisis de entropía de una red}
Uno de los ejercicios solicitaba calcular la entropía de una fuente.
Para ello debemos definir con presicion dos cosas, la \textbf{fuente de informacion} y el \textbf{evento}, para luego calcular su probabilidad
y de allí la entropía.

\newpage

%\section{Resultados}
En esta sección mostraremos los resultados de los experimentos.

\subsection{Experimento 1:}

Las entropías calculadas para las 2 fuentes de información propuestas fueron:

\begin{center}
\begin{tabular}{ l r }
   Fuente& Entropía  \\
\hline
   1 & 1.371641 \\
   2 & 0.570425 \\
 \end{tabular}
\captionof{table}{entropias}
\label{entropiashogar}
\end{center}

Probabilidades para el modelo 1 según IPs:
%\begin{center}
%\begin{tabular}{ l c r }
%   Fuente& IP &probabilidad \\
%\hline
%   1 & 10.0.0.3  &0.864341085271 \\
%   2 & 10.0.0.2& 0.135658914729\\
%\caption{modelo 1}
% \end{tabular}
%\captionof{table}{modelo 1}
%\end{center}

\begin{center}
\begin{tabular}{ l r }
   IP &Probabilidad \\
\hline
192.168.1.138	& 0.3125 \\
192.168.1.103	& 0.0625 \\
192.168.1.137	& 0.1875 \\
192.168.1.1		& 0.3125 \\
192.168.1.143	& 0.125 \\
 \end{tabular}
\captionof{table}{Modelo 1}
\label{probabilidadesModel1}
\end{center}

Probabilidades para el modelo 2 según IPs:
\begin{center}
\begin{tabular}{ l r }
   IP &Probabilidad \\
\hline
192.168.1.1		& 0.9017 \\
192.168.1.103	& 0.0173 \\
192.168.1.143 	& 0.0115 \\
192.168.1.137	& 0.0173 \\
192.168.1.136	& 0.0057 \\
192.168.1.138	& 0.0462 \\
 \end{tabular}
\captionof{table}{Modelo 2}
\label{probabilidadesModel2}
\end{center}

\begin{figure}[H]
	\centering
	\includegraphics[scale = 0.8]{graficos/emiso_is_at_hogar.pdf}
	\caption{Histograma Prob./IPs Modelo 1}
	\label{histogramaprobabilidadesModel1}
\end{figure}


\begin{figure}[H]
	\centering
	\includegraphics[scale = 0.8]{graficos/recep_who_has_hogar.pdf}
	\caption{Histograma Prob./IPs Modelo 2}
	\label{histogramaprobabilidadesModel2}
\end{figure}

\newpage
\subsection{Experimento 2:}

\begin{center}
\begin{tabular}{ l r }
   Fuente& Entropía  \\
\hline
   1 & 3.533684 \\
   2 & 4.352357 \\
 \end{tabular}
\captionof{table}{entropias experimento 2}
\label{entropiasexperimento2}
 %\end{tabular}
\end{center}

En el caso del primer modelo había muchas IP's con la misma probabilidad y datos no significativos que tuvimos que eliminar,
debido a que perdía mucha claridad el gráfico. Pusimos las IP's con probabilidades más significativas.

\begin{figure}[H]
	\centering
	\includegraphics[scale = 0.6]{graficos/emiso_is_at_empresa.pdf}
	\caption{Histograma Prob./IPs Modelo 1}
	\label{histogramaprobabilidadesModel1}
\end{figure}

Con una \textbf{entropía} de \textbf{3.533684}.


En nuestro segundo experimento realizado en una red empresarial (clase B), una breve recolección de datos para el Modelo 2 (receptores de paquetes who-has) arrojó la siguiente información:

\begin{figure}[H]
	\centering
	\includegraphics[scale = 0.6]{graficos/recep_who_has_empresa.pdf}
	\caption{Histograma Prob./IPs Modelo 1}
	\label{histogramaprobabilidadesatos}
\end{figure}

Con una \textbf{entropía} de \textbf{4.352357}.

\section{Envio de paquetes}
En esta sección vamos a ver graficamente la cantidad de paquetes que se envian entre si los nodos de la red,con la esperanza de poder 
sacar algunas conclusiones. Los pesos en las aristas indican la cantidad de paquetes enviados de una IP a otra.

\subsection{Experimento 1}


\begin{figure}[H]
	\centering
	\includegraphics[scale = 0.6]{graficos/emisoras_is_at_red_hogar.png}
	\caption{paquetes is at emitidos en una red hogareña}
      \label{emisorasisat}
\end{figure}




\begin{figure}[H]
	\centering
	\includegraphics[scale = 0.5]{graficos/receptoras_who_has_red_hogar.png}
	\caption{paquetes who has en una red hogareña}
	\label{emisoraswhohas}	
\end{figure}

\newpage


\subsection{Experimento 2}
En este caso como se trataba de una red empresarial tuvimos que tomar recortes del grafo de envío de paquetes, ya que de lo contrario
se tornaba ilegible. Capturamos solo partes que colaboran con el objetivo de recolectar nodos distinguidos. 
Aclaración: En los 2 siguientes gráficos, el nodo apuntado representa a la IP origen mientras que el nodo apuntador es la IP destino.


\begin{figure}[H]
	\centering
	\includegraphics[scale = 0.5]{graficos/receptores_who_has_parte1_atos.png}
	\caption{paquetes who has en una empresarial}
	\label{receptoraswhohasatos}	
\end{figure}


\begin{figure}[H]
	\centering
	\includegraphics[scale = 0.5]{graficos/receptores_who_has_atos_parte2.png}
	\caption{paquetes who has en una red empresarial(segundo nodo)}
	\label{receptoraswhohasatos2 }	
\end{figure}

Aclaración: En los 2 siguientes gráficos, el nodo apuntado representa a la IP destino mientras que el nodo apuntador es la IP origen.

\begin{figure}[H]
	\centering
	\includegraphics[scale = 0.5]{graficos/emisoras_is_at_red_atos.png}
	\caption{paquetes is at en una empresarial}
	\label{emisorasisatatos1 }	
\end{figure}


\begin{figure}[H]
	\centering
	\includegraphics[scale = 0.3]{graficos/emisoras_is_at_atos.png}
	\caption{paquetes is at en una empresarial(segundo nodo)}
	\label{emisorasisatatos2 }
\end{figure}




%\newpage

%\section{Discusion}

En esta sección analizaremos los resultados obtenidos.

\subsection{Analisis de entropias}

En primer experimento definimos dos modelos y calculamos las probabilidades de cada IP y la entropía de las fuentes de información
que definimos. En base a estos resultados y con los gráficos de la sección \textbf{envios de paquetes} vamos a intentar realizar análisis
comparando entropías y viendo como impacta el tipo y tamaño de red en las mismas.

Como podemos ver en el cuadro \ref{probabilidadesModel1} hay dos ip que tienen más probabilidad de ser emisoras de un paquete de tipo is at.
Estos dos nodos de la red son los que de alguna manera más impactan en la entropía ya que al ser sus probabilidades valores significativamente más 
altos, incrementan la esperanza de la información.
Además no hay muchos nodos que son emisores de paquete tipo is at, de hecho si vemos la figura \ref{emisorasisat} podemos apreciar que
el tráfico es coherente con las probabilidades.
 
Ahora si miramos las probabilidades de cada Ip de ser receptora de paquetes who has, vamos a notar en el gráfico \ref{histogramaprobabilidadesModel2}
que la IP 192.168.1.1 resalta, probablemente este nodo en la red es el default gateway(y en efecto lo es).
 La entropía en este caso es mucho menor a la anterior, posiblemente porque la distribución de probabilidades es más
pareja, a excepción del dato recién mencionado. 

Otro análisis que se puede hacer en base a las entropías y probabilidades es que en la figura \ref{histogramaprobabilidadesModel1} 
la entropía es mayor a 1 y las probabilidades son más parejas lo cuál significa que la información de un paquete ARP puede revelar 
mucha más información que en el caso de la figura \ref{histogramaprobabilidadesModel2} donde la entropía es menor a 1 y se
destaca la puerta de enlace predeterminada.

Si hacemos este mismo análisis para la red empresaríal vamos a notar que las probabilidaes son mucho más parejas en todos lo casos, pero
si vemos el gráfico \ref{histogramaprobabilidadesatos} vemos que la IP 172.16.189.14 se destaca. Este nodo podría ser el 
más solicitado por ser una puerta de enlace predeterminada, una base de datos o incluso un recurso disponible y muy requerido.

Por otra parte si observamos la tabla \ref{entropiasexperimento2} y lo comparamos con la tabla \ref{entropiashogar} vamos a poder 
observar que para una misma fuente de información las entropías varían drasticamente dependiendo de la topología de la red y su tamaño.
De hecho en el primer modelo tenemos una entropia de 1,37 contra una de 3.53, y en el segundo modelo,una entropía de 0,57 contra una
de 4.352357.

En cuanto a los valores de la entropías con respecto a las probabilidades, en ambos casos es mayor a 1, lo cual nos indica que
en cada emision/recepción de paquete hay información más valiosa.


%Ahora bien podriamos medir que tanto impacta el tamaño según la proporción.
%En la red hogareña hay solo 7 nodos, mientras que en la red empresarial hay aproximadamente 108 nodos.
%Si definimos como índice \textbf{entropia/cantidad de nodos} podemos ver que para la primer fuente cada nodo aporta un valor 0,19, mientras que 
%en la segunda el valor es de tan solo 0.03, es decir la entropía aportada por cada nodo es mucho menor, posiblemente se deba
%a que las probabilidaes sean más parejas.

%Para la segunda fuente ocurre algo similar pero no tan desproporcioando ya que tenemos un valor aproximado de 0.07 en la hogareña contra 
%un 0.04 de la red empresarial y comparado con el caso anterior, la diferencia no es tan grande.

Como conclusiones generales de todo esto podemos ver que:
\begin{itemize}
 \item El tipo de red impacta en las entropías drasticamente.
 \item En el caso del segundo modelo, el impacto que tiene el tamaño de la red es mucho menor. 
 \item En ambos casos podría haber nodos distinguidos, ya sea por emision o recepción de paquetes.
\item En ambas redes hay muchos nodos con probabilidades de emision/recepcion muy parejas, que sin la presencia de otros nodos distinguidos
  incrementarian la entropía de la red. 
\item La red empresarial es bastante heterogenea o bien tiene fragmentos de la misma destinada a proveer servicios.
\end{itemize}


\section{Nodos distinguidos}
Un nodo distinguido es aquel cuya interacción con otros es más frecuente  ya sea como emisor o receptor de paquetes.
Intentaremos encontrarlos en las dos redes estudiadas y además veremos si los resultados encontrados se relacionan con lo mencionado en
la sección anterior.
Al ser la red hogareña una red pequeña, podemos ver en completitud la cantidad de paquetes enviados y recibidos,
no pudimos hacer lo mismo en la red empresarial ya que el gráfico obtenido era ilegible, como antes mencionamos, decidimos truncarlo cortando la parte que creemos
más importante y mostrando solo aquellos nodos que enviaron y recibieron paquetes de una cantidad más alta de la común.

En la figura \ref{emisorasisat} podemos ver los paquetes is-at emitidos en la red hogareña muchos fueron dirigidos a 192.168.1.138 y
existe cierto nivel de interacción entre 192.168.1.143, 192.168.1.1 y 192.168.1.136.

En la figura \ref{emisoraswhohas} podemos ver una cantidad de paquetes mucho mayor pero con una distribución mucho menos centralizada,
esta vez  se distingue el nodo 192.168.1.136.

En el caso de la red empresarial el análisis fue mucho más dificil de ver a simple vista, hubo que aplicar criterios y tomar muestras
por partes. Intentamos lo siguiente:
\begin{itemize}
\item Tomar fragmentos de la red.
\item Reducir la muestra a un tamaño proporcional(teniendo en cuenta la ditribución de probabilidades).
\item Tomar en cuenta solo aquellos nodo que envian o reciben más de cierta cantidad de paquetes.
\end{itemize} 

De estas tres opciones solo la ultima nos dió resultados y luego allí tomamos una muestra.

En las figuras \ref{receptoraswhohasatos} y \ref{receptoraswhohasatos2 } podemos ver una fuerte interacción con el nodo de IP 
172.16.189.86 y un fenómeno mucho más sorprendente es que muy aisladamente el nodo de IP 172.16.189.1 pregunta por 172.16.189.14
y 172.16.189.103 con mucha frecuencia. Logramos determinar que la 172.16.189.14 no se encontraba disponible (ya sea porque no existía 
en la red o porque el dispositivo se encontraba apagado) y por tal motivo la IP 172.16.189.1 envíaba paquetes ARP who-has periodicamente.
Es probable que suceda algo similar con 172.16.189.103.

%Lo malo de haber truncado el grafo de esta manera es que no podemos ver que efectivamente el nodo de IP 172.16.189.14 es el que más
%probabilidad tiene, posiblemente esto sea porque recibe pocos paquetes por parte de cada uno de los otros nodos, pero si 
%sumamos el total, es el que mas recibe. La misma observación no aplica para el nodo de IP 172.16.189.1 donde claramente se condice con 
%lo mostrado en la figura \ref{histogramaprobabilidadesatos}.

En cuanto a la emisión de paquetes is-at podemos contemplar las figuras \ref{emisorasisatatos1 } y \ref{emisorasisatatos2 }, donde pudimos encontrar
dos casos aislados uno de mayor interacción. En uno  muchos respondian al nodo 172.16.189.167 y en otro una cantidad acotada de nodos
enviaba la misma cantidad de paquetes a 172.16.189.190.

De esto podemos sacar las siguientes conclusiones:
\begin{itemize}
\item Las dos redes presentan nodos aislados a pesar de su distinta utilidad y tamaño, posiblemente se deba a los protocolos de comunicación y armado de tablas de ruteo. 
\item En el caso de la red empresarial hay una relación entre los gráficos de probabilidades de receptores de paquetes who-has y nodos aislados. 
\item La red empresarial presenta muchas más interacciones aisladas, posiblemente tenga una cantidad mayor de dispositivos presten servicios muy específicos.

\end{itemize}

 



 

%\newpage





%\input{conclusiones.tex}	
%\newpage

%\input{apendices.tex}	
%\newpage

%\input{referencias.tex}

\end{document}

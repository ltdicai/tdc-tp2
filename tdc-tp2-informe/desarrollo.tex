\section{Herramientas}

\subsection{Traceroute: idea}

Implementamos nuestra propia herramienta de traceroute siguiendo la técnica del envío de paquetes ICMP ECHO\_REQUEST/TIME\_EXCEEDED/ECHO\_REPLY. Consideramos dos posibles implementaciones: el algoritmo estandard, que consiste en enviar para cada TTL una ráfaga de paquetes; y una modificación en la que se envíe un paquete por TTL hasta alcanzar el host destino o superar el límite de saltos y repetir desde el principio. Elegimos la primera por simplicidad de la implementación al momento de calcular el valor del RTT.

Un punto a considerar cuando se realiza traceroute con ICMP es que cada paquete puede seguir una ruta distinta a la recorrida por los demás (y las rutas  pueden variar incluso entre la ida y la vuelta de un mismo paquete). Por lo tanto, para un TLL dado podríamos obtener respuestas de varios hosts distintos. Para lidiar con este problema, decidimos considerar solo la ruta más problable. Para esto, por cada rafaga de paquetes echo request consideramos como nodo del camino aquel que haya respondido la mayor cantidad de veces (al calcular la frecuencia descartamos los timetouts que hayan sucedido).

Existe otro detalle a resolver una vez que quedan determinados los host del camino: para cada host tenemos una muestra de RTTs que pueden  pueden variar considerablemente. Teniendo en cuenta que el objetivo de nuestra herramienta es estimar un camino con los valores esperados de RTT entre nodos, sopesamos varias alternativas para aplanar los datos. Entre ellas analizamos las siguientes:
\begin{itemize}
\item{Menor RTT}
\item{RTT Promedio}
\item{RTT Promedio, quitando previamente los outliers de la muestra (con el método de Cimbala)}
\item{Mediana de RTT}
\end{itemize}
La herramienta calcula todas ellas a modo de comparación, pero para los análisis nos decidimos por utilizar el RTT promedio pre-filtrado, pues esperamos que resulte en valores significativos que no se vean afectados por datos espurios. 


\subsection{Detección automática de enlaces intercontinentales}

Una vez determinado un camino y los RTT correspondientes, estamos en condiciones de comenzar el análisis para intentar detectar automáticamente los enlaces intercontinentales de larga distancia basandonos en la técnica de estimación de outliers propuesta por Cimbala. Para ello, obtenemos los RTT relativos entre hops consecutivos y aplicamos el algoritmo de Cimbala a fin de detectar outliers. Nuestra hipótesis es que los saltos que tomen mas tiempo, detectados con el metodo de outliers, son los posibles saltos intercontinentales. 

Resulta importante considerar la posibilidad de que algunos hops no tengan definido su RTT. Esto puede suceder cuando, por ejemplo, el hop no implementa ICMP o se encuentra detrás de un firewall que bloqueaba este protocolo. Contemplamos la opción de interpolar estos faltantes, pero concluimos que la falta de información nos posibilita solo a aplicar un interpolado lineal que resultaría en información "suavizada"  que juegue negativamente al momento de aplicar Cimbala. Por esto, decidimos utilizar solamente los hops con RTT definido: si detectamos un posible salto intercontinental y vemos que los hops no son consecutivos, al menos podemos deducir que el salto ocurre entre esos host. 


\subsection{Implementación}

La herramienta fue desarrollada en Python utilizando el paquete \texttt{scapy}, y permite definir los siguientes parámetros de ejecución:
\begin{itemize}
	\item{MAT\_RAFAGA: tamaño de la ráfaga para cada TTL}
	\item{MAX\_TTL: cantidad máxima de saltos esperados}
	\item{TIMEOUT: tiempo de espera, medido en segundos }
	\item{P (Tolerancia a timeouts): cantidad de timeouts seguidos que se toleran antes de decidir que no hay respuesta, medido en porcentaje del tamaño de la ráfaga.}
	\item{OUTPUT: identificador para generar los nombres de los archivos de salida}
\end{itemize}

El código se divide en tres funciones principales
\begin{itemize}
	\item{\texttt{rastrear()}: es la implementacion del algoritmo de traceroute utilizando paquetes ICMP. Devuelve un muestreo de RTTs por TTL.}
	\item{\texttt{generar\_camino()}: a partir del muestreo devuelto por 'rastrear', decide cual es el camino más probable y calcula los RTT correspondientes a cada hop.}
	\item{\texttt{detectar\_enlaces\_intercontinentales()}: dato un camino devuelto por \texttt{generar\_camino()}, calcula los RTT entre hops y marca aquellos que pueden ser saltos intercontinentales mediante la detección de outlier según el algoritmo de Cimbala.}
\end{itemize}	

Para el mapeo de IP a País utilizamos una base de datos obtenida de \href{http://www.maxmind.com}{MaxMind} actualizada al 09/07/2016, accedida en el código por medio del paquete \texttt{geoip2} de Python. Descartamos otras fuentes de datos que debido a que presentaban limitaciones de performance, no proveían una API o se encontraban desactualizadas. 


\section{Experimentos}
A fin de probar el comportamiento del algoritmo propuesto, recolectamos los resultados de ejecutar el programa tomando como destinos a cuatro universidades en distintos continentes. Realizamos 3 corridas para cada destino, variando el tamaño de la ráfaga en 50, 150 y 300 paquetes.
Las destinos elegidos fueron:

\begin{center}
   \begin{tabular}{ | c | c | c | c | }
     \hline
     \textbf{Universidad} & \textbf{Host} & \textbf{Pais} & \textbf{Contiente} \\ \hline
     Universidad de Oregon & \url{www.cs.uoregon.edu} & Estados Unidos & América del Norte\\ \hline
     universidad de Tokio & \url{www.u-tokyo.ac.jp} & Japón & Asia\\ \hline
     Universidad de Moscú & \url{msu.ru} & Rusia & Europa del Este\\ \hline
     universidad de Sudáfrica & \url{www.unisa.ac.za} & Sudáfrica & África \\ \hline
   \end{tabular}
 \end{center}
 
 
\begin{figure}[H]
  \centering
  \includegraphics[scale = 0.3]{imagenes/mapa.png}
  \caption{Ubicación de las universidades elegidas}
  \label{histogramaprobabilidadesModel1}
\end{figure}


Los experimentos se realizaron utilizando en una computadora con Linux Mint conectada a Internet (provisto por Fibertel) por medio de un enlace WiFi. Se utilizaron los siguientes comandos (donde $n$ es el tamaño de la ráfaga) 
\begin{table}[]
\centering
\caption{Saltos y sus RTT en el camino desde Buenos Aires a la Universidad de Sudáfrica (África).}
%\label{Universidad}
\begin{tabular}{ | c | c | l |}
	\hline 
Exp & Destino & Comando\\ \hline
 1 & Sudáfrica & \texttt{sudo python traceroute.py www.unisa.ac.za -tr n -p 0.3 -m 30 -o AFR\_n} \\ 
  2 & EEUU& \texttt{sudo python traceroute.py www.cs.uoregon.edu -tr n -p 0.3 -m 30 -o EEUU\_n} \\ 
  3 & Japon& \texttt{sudo python traceroute.py www.u-tokyo.ac.jp -tr n -p 0.3 -m 30 -o JAP\_n} \\ 
  4 & Rusia & \texttt{sudo python traceroute.py msu.ru -tr n -p 0.3 -m 30 -o RUSIA\_n} \\ 
  \hline

\end{tabular}
\end{table}
  Notar que la herramienta debe ejecutarse con permisos de \texttt{root}.
  
\section{Resultados}

En las siguientes tablas mostramos los resultados obtenidos para distintos tamaños de ráfagas. Notar que solo estamos utilizando el RTT promedio con los outliers previamente eliminados.

\newpage
\section{Experimento 1: África}


\begin{table}[]
\centering
\caption{Camino estimado desde Buenos Aires a la Universidad de Sudáfrica (África).}
%\label{Universidad}
\begin{tabular}{ | c | c | c | c | c | c | }
	\hline 
  & & & Rafaga = 50 & Rafaga = 150 &Rafaga = 300  \\ %\hline
  TTL	&         IP        & 	        Pais      &  	 RTT Filtrado Prom  &	 RTT Filtrado Prom  & RTT Filtrado Prom  \\ \hline  
  1	&192.168.0.1        & 	       Local      &  	      63.183 ms     &	       63.66 ms     &      60.306 ms     \\ \hline  
  2	&None               & 	      Unknown     &  	      Unknown       &	      Unknown       &      Unknown       \\ \hline  
  3	&None               & 	      Unknown     &  	      Unknown       &	      Unknown       &      Unknown       \\ \hline  
  4	&None               & 	      Unknown     &  	      Unknown       &	      Unknown       &      Unknown       \\ \hline  
  5	&None               & 	      Unknown     &  	      Unknown       &	      Unknown       &      Unknown       \\ \hline  
  6	&200.89.164.129     & 	     Argentina    &  	      75.735 ms     &	      69.777 ms     &      72.186 ms     \\ \hline  
  7	&200.89.165.130     & 	     Argentina    &  	        81.6 ms     &	      69.322 ms     &      75.228 ms     \\ \hline  
  8	&200.89.165.222     & 	     Argentina    &  	      82.939 ms     &	      69.796 ms     &      74.685 ms     \\ \hline  
  9	&195.22.220.172     & 	       Italy      &  	      72.147 ms     &	       66.17 ms     &      72.982 ms     \\ \hline  
  10	&89.221.41.161      & 	       Italy      &  	     201.218 ms     &	     197.541 ms     &     202.749 ms     \\ \hline  
  11	&89.221.41.161      & 	       Italy      &  	     199.258 ms     &	      195.86 ms     &     203.352 ms     \\ \hline  
  12	&154.54.9.17        & 	   United States  &  	     201.047 ms     &	     209.585 ms     &     203.779 ms     \\ \hline  
  13	&154.54.24.233      & 	   United States  &  	     198.495 ms     &	     201.588 ms     &      197.17 ms     \\ \hline  
  14	&154.54.24.197      & 	   United States  &  	     219.718 ms     &	     228.412 ms     &     211.934 ms     \\ \hline  
  15	&154.54.24.221      & 	   United States  &  	     224.876 ms     &	       225.9 ms     &     221.758 ms     \\ \hline  
  16	&154.54.40.109      & 	   United States  &  	     241.303 ms     &	     231.854 ms     &     228.356 ms     \\ \hline  
  17	&154.54.42.86       & 	   United States  &  	     294.135 ms     &	     288.052 ms     &     287.978 ms     \\ \hline  
  18	&154.54.58.186      & 	   United States  &  	     298.166 ms     &	     289.808 ms     &     287.868 ms     \\ \hline  
  19	&154.54.56.238      & 	   United States  &  	     297.178 ms     &	     289.593 ms     &     287.225 ms     \\ \hline  
  20	&149.14.80.210      & 	   United States  &  	      286.71 ms     &	     289.823 ms     &     287.583 ms     \\ \hline  
  21	&196.32.209.174     & 	    South Africa  &  	     472.497 ms     &	      469.31 ms     &     466.299 ms     \\ \hline  
  22	&155.232.6.65       & 	    South Africa  &  	      480.99 ms     &	     471.961 ms     &      467.52 ms     \\ \hline  
  23	&155.232.6.37       & 	    South Africa  &  	     470.571 ms     &	     469.022 ms     &     465.979 ms     \\ \hline  
  24	&155.232.6.33       & 	    South Africa  &  	      475.67 ms     &	     472.859 ms     &     470.095 ms     \\ \hline  
  25	&155.232.6.142      & 	    South Africa  &  	     471.279 ms     &	     475.578 ms     &     464.099 ms     \\ \hline  
  26	&155.232.6.145      & 	    South Africa  &  	     498.626 ms     &	     493.943 ms     &     498.355 ms     \\ \hline  
  27	&155.232.6.138      & 	    South Africa  &  	     463.824 ms     &	     473.392 ms     &     511.219 ms     \\ \hline  
  28	&None               & 	      Unknown     &  	      Unknown       &	      Unknown       &      Unknown       \\ \hline  
  29	&None               & 	      Unknown     &  	      Unknown       &	      Unknown       &      Unknown       \\ \hline  
  30	&None               & 	      Unknown     &  	      Unknown       &	      Unknown       &      Unknown       \\ \hline  
\end{tabular}
\end{table}


\begin{landscape}
\begin{table}[]
\centering
\caption{Saltos y sus RTT en el camino desde Buenos Aires a la Universidad de Sudáfrica (África).}
%\label{Universidad}
\begin{tabular}{ | c | c | c | c | c | c | }
	\hline 
Salto	& 50	& 150	& 300 \\ \hline
               Origen                ->               Destino              & 	Promedio Filtrado&	Promedio Filtrado&	Promedio Filtrado\\ \hline
( 1) 192.168.0.1     Local           -> ( 6) 200.89.164.129  Argentina     & 	    12.552 ms   	&  [  6.118 ms]  	 & [  11.88 ms]  \\ \hline
( 6) 200.89.164.129  Argentina       -> ( 7) 200.89.165.130  Argentina     & 	     5.865 ms   	&     0.456 ms   	 &    3.043 ms   \\ \hline
( 7) 200.89.165.130  Argentina       -> ( 8) 200.89.165.222  Argentina     & 	     1.339 ms   	&     0.474 ms   	 &    0.544 ms   \\ \hline
( 8) 200.89.165.222  Argentina       -> ( 9) 195.22.220.172  Italy         & 	    10.792 ms   	&     3.626 ms   	 &    1.702 ms   \\ \hline
( 9) 195.22.220.172  Italy           -> (10) 89.221.41.161   Italy         & 	  [129.071 ms]  	&  [131.371 ms]  	 & [129.767 ms]  \\ \hline
(10) 89.221.41.161   Italy           -> (11) 89.221.41.161   Italy         & 	     1.961 ms   	&      1.68 ms   	 &    0.603 ms   \\ \hline
(11) 89.221.41.161   Italy           -> (12) 154.54.9.17     United States & 	     1.789 ms   	&  [ 13.724 ms]  	 &    0.427 ms   \\ \hline
(12) 154.54.9.17     United States   -> (13) 154.54.24.233   United States & 	     2.552 ms   	&  [  7.997 ms]  	 &    6.609 ms   \\ \hline
(13) 154.54.24.233   United States   -> (14) 154.54.24.197   United States & 	  [ 21.222 ms]  	&  [ 26.824 ms]  	 & [ 14.764 ms]  \\ \hline
(14) 154.54.24.197   United States   -> (15) 154.54.24.221   United States & 	     5.158 ms   	&     2.512 ms   	 & [  9.824 ms]  \\ \hline
(15) 154.54.24.221   United States   -> (16) 154.54.40.109   United States & 	  [ 16.428 ms]  	&  [  5.954 ms]  	 &    6.598 ms   \\ \hline
(16) 154.54.40.109   United States   -> (17) 154.54.42.86    United States & 	  [ 52.831 ms]  	&  [ 56.199 ms]  	 & [ 59.622 ms]  \\ \hline
(17) 154.54.42.86    United States   -> (18) 154.54.58.186   United States & 	     4.031 ms   	&     1.755 ms   	 &    0.111 ms   \\ \hline
(18) 154.54.58.186   United States   -> (19) 154.54.56.238   United States & 	     0.988 ms   	&     0.215 ms   	 &    0.643 ms   \\ \hline
(19) 154.54.56.238   United States   -> (20) 149.14.80.210   United States & 	    10.468 ms   	&      0.23 ms   	 &    0.359 ms   \\ \hline
(20) 149.14.80.210   United States   -> (21) 196.32.209.174  South Africa  & 	  [185.787 ms]  	&  [179.487 ms]  	 & [178.716 ms]  \\ \hline
(21) 196.32.209.174  South Africa    -> (22) 155.232.6.65    South Africa  & 	     8.493 ms   	&     2.652 ms   	 &     1.22 ms   \\ \hline
(22) 155.232.6.65    South Africa    -> (23) 155.232.6.37    South Africa  & 	    10.419 ms   	&      2.94 ms   	 &     1.54 ms   \\ \hline
(23) 155.232.6.37    South Africa    -> (24) 155.232.6.33    South Africa  & 	     5.099 ms   	&     3.838 ms   	 &    4.115 ms   \\ \hline
(24) 155.232.6.33    South Africa    -> (25) 155.232.6.142   South Africa  & 	     4.391 ms   	&     2.719 ms   	 &    5.996 ms   \\ \hline
(25) 155.232.6.142   South Africa    -> (26) 155.232.6.145   South Africa  & 	  [ 27.347 ms]  	&  [ 18.365 ms]  	 & [ 34.256 ms]  \\ \hline
(26) 155.232.6.145   South Africa    -> (27) 155.232.6.138   South Africa  & 	  [ 34.802 ms]  	&  [ 20.551 ms]  	 & [ 12.863 ms]  \\ \hline
\end{tabular}
\end{table}

\end{landscape}

\begin{figure}[H]
  \centering
  %\includegraphics[scale = 0.8]{imagenes/africaTTL.png}
  \caption{RTT estimado del traceroute a la Universidad de Sudáfrica}
  \label{africaTTL}
  \subfigure[RTT promedio por TTL]{\includegraphics[scale = 1]{imagenes/africaTTL.png} }
  \subfigure[RTT relativos entre saltos]{\includegraphics[scale = 1]{imagenes/africaRTTrelativos.png}}
\end{figure}

\newpage
\subsection{Experimento 2: EEUU}
\begin{table}[H]
\centering
\caption{Camino estimado desde Buenos Aires a la Universidad de Oregon (America del Norte).}
%\label{Universidad}
\begin{tabular}{ | c | c | c | c | c | c | }
	\hline 
  & & & 50	& 150	& 300 \\ %\hline
TTL	&          IP         	     &    Pais        & 	 RTT Filtrado Prom  	&  RTT Filtrado Prom  & 	 RTT Filtrado Prom  \\ \hline  
1	& 192.168.0.1       &   	       Local      &   	       99.53 ms    &  	     103.514 ms   &   	      57.461 ms   \\ \hline    
2	& None               &  	      Unknown     &   	      Unknown      &  	      Unknown     &   	      Unknown     \\ \hline    
3	& None               &  	      Unknown     &   	      Unknown      &  	      Unknown     &   	      Unknown     \\ \hline    
4	& None               &  	      Unknown     &   	      Unknown      &  	      Unknown     &   	      Unknown     \\ \hline    
5	& None               &  	      Unknown     &   	      Unknown      &  	      Unknown     &   	      Unknown     \\ \hline    
6	& 200.89.165.141    &   	     Argentina    &   	     106.059 ms    &  	     102.341 ms   &   	      78.394 ms   \\ \hline    
7	& 200.89.165.130    &   	     Argentina    &   	     101.341 ms    &  	     103.905 ms   &   	      71.902 ms   \\ \hline    
8	& 200.89.165.222    &   	     Argentina    &   	     104.182 ms    &  	     103.432 ms   &   	      77.629 ms   \\ \hline    
9	& 190.216.88.33     &   	     Argentina    &   	     107.429 ms    &  	     101.819 ms   &   	      79.446 ms   \\ \hline    
10	& 67.17.94.249     &    	   United States  &   	     308.066 ms    &  	     303.806 ms   &   	     202.742 ms   \\ \hline    
11	& None               &  	      Unknown     &   	      Unknown      &  	      Unknown     &   	     783.798 ms   \\ \hline    
12	& 4.69.132.149    &     	   United States  &   	     367.776 ms    &  	     305.802 ms   &   	     246.129 ms   \\ \hline    
13	& 4.69.132.149    &     	   United States  &   	     307.944 ms    &  	     307.181 ms   &   	     247.065 ms   \\ \hline    
14	& 4.53.200.2      &     	   United States  &   	     299.937 ms    &  	     309.681 ms   &   	     261.722 ms   \\ \hline    
15	& 207.98.64.165     &   	   United States  &   	     306.661 ms    &  	     305.197 ms   &   	     258.771 ms   \\ \hline    
16	& 207.98.68.182     &   	   United States  &   	     313.137 ms    &  	     306.078 ms   &   	     261.577 ms   \\ \hline    
17	& 128.223.2.1       &   	   United States  &   	     303.678 ms    &  	     308.079 ms   &   	     261.194 ms   \\ \hline    
18	& 128.223.4.25      &   	   United States  &   	     306.301 ms    &  	     306.542 ms   &   	      257.81 ms   \\ \hline    
\end{tabular}
\end{table}

\begin{landscape}
\begin{table}[]
\centering
\caption{Saltos y sus RTT en el camino desde Buenos Aires a la Universidad de Sudáfrica (África).}
%\label{Universidad}
\begin{tabular}{ | c | c | c | c | c | c | }
	\hline 
Salto	& 50	& 150	& 300 \\ \hline
               Origen                ->               Destino               	&Promedio Filtrado&	Promedio Filtrado&	Promedio Filtrado \\ \hline
( 1) 192.168.0.1     Local           -> ( 6) 200.89.165.141  Argentina     & 	     6.528 ms  & 	     1.173 ms   &	  [ 20.933 ms]   \\ \hline
( 6) 200.89.165.141  Argentina       -> ( 7) 200.89.165.130  Argentina     & 	     4.718 ms  & 	     1.564 ms   &	     6.493 ms    \\ \hline
( 7) 200.89.165.130  Argentina       -> ( 8) 200.89.165.222  Argentina     & 	     2.841 ms  & 	     0.472 ms   &	     5.727 ms    \\ \hline
( 8) 200.89.165.222  Argentina       -> ( 9) 190.216.88.33   Argentina     & 	     3.247 ms  & 	     1.613 ms   &	     1.817 ms    \\ \hline
( 9) 190.216.88.33   Argentina       -> (10) 67.17.94.249    United States & 	  [200.638 ms] & 	  [201.987 ms]  &	  [123.297 ms]   \\ \hline
(10) 67.17.94.249    United States   -> (12) 4.69.132.149    United States  &	  [  59.71 ms]  &	     1.996 ms    &     	x \\ \hline
(10) 67.17.94.249    United States   -> (11) 4.68.72.66      United States   &       	x      	 &           x	     &     [581.056 ms]   \\ \hline
(11) 4.68.72.66      United States   -> (12) 4.69.132.149    United States  &	       x        &   	       x	     &     [537.669 ms]   \\ \hline
(12) 4.69.132.149    United States   -> (13) 4.69.132.149    United States & 	  [ 59.833 ms] & 	     1.378 ms   &	     0.935 ms    \\ \hline
(13) 4.69.132.149    United States   -> (14) 4.53.200.2      United States & 	     8.006 ms  & 	       2.5 ms   &	  [ 14.657 ms]   \\ \hline
(14) 4.53.200.2      United States   -> (15) 207.98.64.165   United States & 	     6.724 ms  & 	  [  4.484 ms]  &	     2.951 ms    \\ \hline
(15) 207.98.64.165   United States   -> (16) 207.98.68.182   United States & 	     6.477 ms  & 	     0.881 ms   &	     2.807 ms    \\ \hline
(16) 207.98.68.182   United States   -> (17) 128.223.2.1     United States & 	      9.46 ms  & 	     2.001 ms   &	     0.384 ms    \\ \hline
(17) 128.223.2.1     United States   -> (18) 128.223.4.25    United States & 	     2.623 ms  & 	     1.536 ms   &	     3.384 ms    \\ \hline

\end{tabular}
\end{table}

\end{landscape}

\begin{figure}[H]
  \centering
  %\includegraphics[scale = 0.8]{imagenes/eeuuTTL.png}
  \caption{RTT estimado del traceroute a la Universidad de Oregon }
  \label{eeuuTTL}
  \subfigure[RTT promedio por TTL]{\includegraphics[scale = 1]{imagenes/eeuuTTL.png} }
  \subfigure[RTT relativos entre saltos]{\includegraphics[scale = 1]{imagenes/eeuuRTTrelativos.png}}
\end{figure}


\newpage
\subsection{Experimento 3: Japon}

\begin{table}[]
\centering
\caption{Camino estimado desde Buenos Aires a la Universidad de Tokio (Asia).}
%\label{Universidad}
\begin{tabular}{ | c | c | c | c | c | c | }
\hline
 &&& 50 & 150 & 300 \\ %\hline
TTL	&         IP     &    	        Pais     &   	 RTT Filtrado Prom & 	 RTT Filtrado Prom  &	 RTT Filtrado Prom \\ \hline  
1	&192.168.0.1     &    	       Local     &   	      59.684 ms    & 	       62.26 ms     &	      65.531 ms  \\ \hline    
2	&None            &    	      Unknown    &   	      Unknown      & 	      Unknown       &	      Unknown    \\ \hline    
3	&None            &    	      Unknown    &   	      Unknown      & 	      Unknown       &	      Unknown    \\ \hline    
4	&None            &    	      Unknown    &   	      Unknown      & 	      Unknown       &	      Unknown    \\ \hline    
5	&None            &    	      Unknown    &   	      Unknown      & 	      Unknown       &	      Unknown    \\ \hline    
6	&200.89.164.129  &    	     Argentina   &   	      66.602 ms    & 	      80.867 ms     &	      76.751 ms  \\ \hline    
7	&200.89.165.5    &    	     Argentina   &   	      65.983 ms    & 	      85.396 ms     &	      77.617 ms  \\ \hline    
8	&200.89.165.250  &    	     Argentina   &   	      78.198 ms    & 	      81.414 ms     &	      79.964 ms  \\ \hline    
9	&185.70.203.56   &    	       Italy     &   	      71.299 ms    & 	      88.681 ms     &	      75.557 ms  \\ \hline    
10	&195.22.219.3    &    	       Italy     &   	     105.539 ms    & 	      105.86 ms     &	      98.037 ms  \\ \hline    
11	&195.22.219.3    &    	       Italy     &   	       103.2 ms    & 	     100.989 ms     &	      99.468 ms  \\ \hline    
12	&149.3.181.65    &    	       Italy     &   	     213.327 ms    & 	     217.664 ms     &	     209.684 ms  \\ \hline    
13	&129.250.2.227   &    	   United States &   	      255.97 ms    & 	     261.183 ms     &	     254.077 ms  \\ \hline    
14	&129.250.4.13    &    	   United States &   	     297.467 ms    & 	     310.539 ms     &	     303.424 ms  \\ \hline    
15	&129.250.2.54    &    	   United States &   	     298.115 ms    & 	     307.844 ms     &	     303.238 ms  \\ \hline    
16	&129.250.3.86    &    	   United States &   	     403.231 ms    & 	     418.358 ms     &	     406.681 ms  \\ \hline    
17	&129.250.6.188   &    	   United States &   	     396.553 ms    & 	     416.043 ms     &	     400.489 ms  \\ \hline    
18	&129.250.2.255   &    	   United States &   	     394.784 ms    & 	     404.056 ms     &	     398.441 ms  \\ \hline    
19	&61.200.80.218   &    	       Japan     &   	     392.899 ms    & 	     409.514 ms     &	      391.58 ms  \\ \hline    
20	&158205192173	 &      Japan        	  &   391.804 ms     	    & 412.988 ms     	     &393.182 ms     \\ \hline    
21	&158.205.192.86  &    	       Japan     &   	     425.468 ms    & 	     438.572 ms     &	     426.033 ms  \\ \hline     
22	&158205121250	 &      Japan        	  &   401.164 ms     	    &  424.37 ms     	     &405.547 ms     \\ \hline    
23	&154.34.240.254  &    	       Japan     &   	     400.949 ms    & 	     415.033 ms     &	       572.7 ms  \\ \hline    
24	&210152135178	 &      Japan        	  &    397.01 ms     	    & 416.837 ms     	     &401.609 ms         \\ \hline

\end{tabular}
\end{table}
% 
\begin{landscape}
  \begin{table}[]
\centering
\caption{Saltos y sus RTT en el camino desde Buenos Aires a la Universidad de Sudáfrica (África).}
%\label{Universidad}
\begin{tabular}{ | c | c | c | c | c | c | }
	\hline 
Salto	& 50	& 150	& 300 \\ \hline
Origen                ->               Destino               	               & Promedio Filtrado &  Promedio Filtrado&	Promedio Filtrado \\ \hline
( 1) 192.168.0.1     Local           -> ( 6) 200.89.164.129  Argentina    &  	     6.918 ms   &	  [ 18.606 ms]  	 & [  11.22 ms]   \\ \hline
( 6) 200.89.164.129  Argentina       -> ( 7) 200.89.165.5    Argentina    &  	     0.619 ms   &	     4.529 ms   	 &    0.867 ms    \\ \hline
( 7) 200.89.165.5    Argentina       -> ( 8) 200.89.165.250  Argentina    &  	  [ 12.215 ms]  &	     3.982 ms   	 &    2.347 ms    \\ \hline
( 8) 200.89.165.250  Argentina       -> ( 9) 185.70.203.56   Italy        &  	       6.9 ms   &	  [  7.267 ms]  	 &    4.407 ms    \\ \hline
( 9) 185.70.203.56   Italy           -> (10) 195.22.219.3    Italy        &  	  [ 34.241 ms]  &	  [ 17.178 ms]  	 & [  22.48 ms]   \\ \hline
(10) 195.22.219.3    Italy           -> (11) 195.22.219.3    Italy        &  	     2.339 ms   &	      4.87 ms   	 &    1.431 ms    \\ \hline
(11) 195.22.219.3    Italy           -> (12) 149.3.181.65    Italy        &  	  [110.127 ms]  &	  [116.675 ms]  	 & [110.216 ms]   \\ \hline
(12) 149.3.181.65    Italy           -> (13) 129.250.2.227   United States&  	  [ 42.643 ms]  &	  [ 43.519 ms]  	 & [ 44.393 ms]   \\ \hline
(13) 129.250.2.227   United States   -> (14) 129.250.4.13    United States&  	  [ 41.497 ms]  &	  [ 49.356 ms]  	 & [ 49.347 ms]   \\ \hline
(14) 129.250.4.13    United States   -> (15) 129.250.2.54    United States&  	     0.648 ms   &	     2.695 ms   	 &    0.186 ms    \\ \hline
(15) 129.250.2.54    United States   -> (16) 129.250.3.86    United States&  	  [105.116 ms]  &	  [110.514 ms]  	 & [103.443 ms]   \\ \hline
(16) 129.250.3.86    United States   -> (17) 129.250.6.188   United States&  	     6.678 ms   &	     2.315 ms   	 &    6.192 ms    \\ \hline
(17) 129.250.6.188   United States   -> (18) 129.250.2.255   United States&  	     1.769 ms   &	  [ 11.987 ms]  	 &    2.048 ms    \\ \hline
(18) 129.250.2.255   United States   -> (19) 61.200.80.218   Japan        &  	     1.885 ms   &	     5.458 ms   	 &    6.861 ms    \\ \hline
(19) 61.200.80.218   Japan           -> (20) 158.205.192.173 Japan        &  	     1.095 ms   &	     3.474 ms   	 &    1.603 ms    \\ \hline
(20) 158.205.192.173 Japan           -> (21) 158.205.192.86  Japan        &  	  [ 33.664 ms]  &	  [ 25.584 ms]  	 & [  32.85 ms]   \\ \hline
(21) 158.205.192.86  Japan           -> (22) 158.205.121.250 Japan        &  	  [ 24.304 ms]  &	  [ 14.202 ms]  	 & [ 20.485 ms]   \\ \hline
(22) 158.205.121.250 Japan           -> (23) 154.34.240.254  Japan        &  	     0.215 ms   &	  [  9.337 ms]  	 & [167.152 ms]   \\ \hline
(23) 154.34.240.254  Japan           -> (24) 210.152.135.178 Japan        &  	     3.939 ms   &	     1.805 ms   	 & [171.091 ms]   \\ \hline
\end{tabular}
\end{table}

\end{landscape}

\begin{figure}[H]
  \centering
  %\includegraphics[scale = 0.8]{imagenes/japonTTL.png}
  \caption{RTT promedio de traceroute a la Universidad de Tokio }
  \label{japonTTL}
  \subfigure[RTT promedio por TTL]{\includegraphics[scale = 1]{imagenes/japonTTL.png} }
  \subfigure[RTT relativos entre saltos]{\includegraphics[scale = 1]{imagenes/japonRTTrelativos.png}}
\end{figure}


\newpage
\subsection{Experimento 4: Rusia}
\begin{table}[]
\centering
\caption{Camino estimado desde Buenos Aires a la Universidad de Mosvú (Europa del este).}
%\label{Universidad}
\begin{tabular}{ | c | c | c | c | c | c | }
\hline
 & & & 50 & 150 & 300 \\ %\hline
TTL	&         IP    &     	        Pais     &   	 RTT Filtrado Prom&  	 RTT Filtrado Prom&  	 RTT Filtrado Prom  \\ \hline
1	&192.168.0.1    &     	       Local     &   	      66.082 ms   &  	     101.009 ms   &  	      62.141 ms \\ \hline    
2	&None           &     	      Unknown    &   	      Unknown     &  	      Unknown     &  	      Unknown   \\ \hline    
3	&None           &     	      Unknown    &   	      Unknown     &  	      Unknown     &  	      Unknown   \\ \hline    
4	&None           &     	      Unknown    &   	      Unknown     &  	      Unknown     &  	      Unknown   \\ \hline    
5	&None           &     	      Unknown    &   	      Unknown     &  	      Unknown     &  	      Unknown   \\ \hline    
6	&200.89.164.141 &     	     Argentina   &   	      67.789 ms   &  	     103.426 ms   &  	      74.055 ms \\ \hline    
7	&200.89.165.130 &     	     Argentina   &   	       63.79 ms   &  	     101.435 ms   &  	      76.409 ms \\ \hline    
8	&200.89.165.222 &     	     Argentina   &   	      64.843 ms   &  	     103.571 ms   &  	       83.03 ms \\ \hline    
9	&190.216.88.33  &     	     Argentina   &   	      68.088 ms   &  	     101.673 ms   &  	      73.911 ms \\ \hline    
10	&67.17.99.233   &     	   United States &   	     237.063 ms   &  	     306.126 ms   &  	     203.308 ms \\ \hline    
11	&None           &     	      Unknown    &   	      Unknown     &  	    1027.876 ms   &  	     980.007 ms \\ \hline    
12	&4.69.158.253   &     	   United States &   	      326.02 ms   &  	     407.603 ms   &  	     327.569 ms \\ \hline    
13	&4.69.158.253   &     	   United States &   	     325.194 ms   &  	     408.104 ms   &  	      327.74 ms \\ \hline    
14	&213242110198	  & United Kingdom   	    & 325.508 ms     	     &408.248 ms     	     &323.149 ms     \\ \hline
15	&None           &     	      Unknown    &   	      Unknown     &  	      Unknown     &  	      Unknown   \\ \hline    
16	&194.85.40.229  &     	       Russia    &   	      338.62 ms   &  	     410.191 ms   &  	     336.288 ms \\ \hline    
17	&194190254118	  &     Russia       	    & 336.555 ms     	     &410.227 ms     	     &335.396 ms     \\ \hline
18	&93.180.0.172   &     	       Russia    &   	     339.032 ms   &  	     409.307 ms   &  	     335.718 ms \\ \hline    
19	&188.44.33.30   &     	       Russia    &   	     341.636 ms   &  	     409.164 ms   &  	     337.758 ms \\ \hline    
20	&188.44.33.2    &     	       Russia    &   	     346.137 ms   &  	      408.97 ms   &  	     343.093 ms \\ \hline    
21	&188.44.50.103  &     	       Russia    &   	     340.466 ms   &  	     410.042 ms   &  	     333.481 ms \\ \hline     

\end{tabular}
\end{table}
\begin{landscape}
\begin{table}[]
\centering
\caption{Saltos y sus RTT en el camino desde Buenos Aires a la Universidad de Moscú (Europa del Este).}
%\label{Universidad}
\begin{tabular}{ | c | c | c | c | }
	\hline 
Salto	& 50	& 150	& 300 \\ \hline
Origen                ->               Destino               	                 & Promedio Filtrado	 & Promedio Filtrado&	Promedio Filtrado\\ \hline
( 1) 192.168.0.1     Local           -> ( 6) 200.89.164.141  Argentina     & 	     1.707 ms   	 &    2.416 ms   	  &  11.913 ms   \\ \hline
( 6) 200.89.164.141  Argentina       -> ( 7) 200.89.165.130  Argentina     & 	     3.999 ms   	 &     1.99 ms   	  &   2.354 ms   \\ \hline
( 7) 200.89.165.130  Argentina       -> ( 8) 200.89.165.222  Argentina     & 	     1.053 ms   	 &    2.136 ms   	  &   6.621 ms   \\ \hline
( 8) 200.89.165.222  Argentina       -> ( 9) 190.216.88.33   Argentina     & 	     3.245 ms   	 &    1.898 ms   	  &   9.119 ms   \\ \hline
( 9) 190.216.88.33   Argentina       -> (10) 67.17.99.233    United States & 	  [168.974 ms]  	 & [204.453 ms]  	  &[129.397 ms]  \\ \hline
(10) 67.17.99.233    United States   -> (12) 4.69.158.253    United States  	  & [ 88.958 ms]       &      	x	         &    x\\ \hline
(10) 67.17.99.233    United States   -> (11) 4.68.72.66      United States & 	     x                 &  [ 721.75 ms]  &	  [776.699 ms]  \\ \hline
(11) 4.68.72.66      United States   -> (12) 4.69.158.253    United States & 	       x               &  [620.273 ms]  &	  [652.438 ms]  \\ \hline
(12) 4.69.158.253    United States   -> (13) 4.69.158.253    United States & 	     0.827 ms   	 &    0.502 ms   	  &   0.172 ms   \\ \hline
(13) 4.69.158.253    United States   -> (14) 213.242.110.198 United Kingdom& 	     0.314 ms   	 &    0.143 ms   	  &   4.592 ms   \\ \hline
(14) 213.242.110.198 United Kingdom  -> (16) 194.85.40.229   Russia        & 	  [ 13.113 ms]  	 &    1.943 ms   	  &  13.139 ms   \\ \hline
(16) 194.85.40.229   Russia          -> (17) 194.190.254.118 Russia        & 	     2.065 ms   	 &    0.035 ms   	  &   0.893 ms   \\ \hline
(17) 194.190.254.118 Russia          -> (18) 93.180.0.172    Russia        & 	     2.477 ms   	 &    0.919 ms   	  &   0.323 ms   \\ \hline
(18) 93.180.0.172    Russia          -> (19) 188.44.33.30    Russia        & 	     2.604 ms   	 &    0.143 ms   	  &    2.04 ms   \\ \hline
(19) 188.44.33.30    Russia          -> (20) 188.44.33.2     Russia        & 	     4.501 ms   	 &    0.195 ms   	  &   5.335 ms   \\ \hline
(20) 188.44.33.2     Russia          -> (21) 188.44.50.103   Russia        & 	  [  5.671 ms]  	 &    1.072 ms   	  &   9.612 ms   \\ \hline

\end{tabular}
\end{table}

\end{landscape}

\begin{figure}[H]
  \centering
  %\includegraphics[scale = 0.8]{imagenes/rusiaTTL.png}
  \caption{RTT promedio del traceroute a la Universidad de Moscú}
  \label{rusiaTTL}
  \subfigure[RTT promedio por TTL]{\includegraphics[scale = 1]{imagenes/rusiaTTL.png} }
  \subfigure[RTT relativos entre saltos]{\includegraphics[scale = 1]{imagenes/rusiaRTTrelativos.png}}
\end{figure}


\section{Análisis de los resultados}


Si bien todos los tamaños de ráfagas devuelven caminos con igual cantidad de hops, a veces algunos host no llegan a responder. Por ejemplo, el host con IP 4.68.72.66 (hop 11 tanto del camino a EEUU como a Rusia) parece no poder detectarse a menos que la ráfaga sea grande. Suponemos que el umbral de timeout que estamos utilizamos puede ser demasiado bajo, por lo que no se alcanaza a recibir una respuesta del host. Ahora, puede ser que el motivo por el cual este host no responda y sucedan timeouts suficientes como para descartar el hop sea que se están siguiendo caminos alternativos que no pasan por él, sino por otro nodo que ignora los paquetes ECHO\_REQUEST. 

También observamos la ocurrencia de comportamientos anómalos. Por ejemplo, en el camino a Rusia podemos ver que los hops 12 y 13 corresponden a la misma IP. Esto puede deberse a que se atraviesa una ruta MPLS que delega en ese host la tarea de enviar los mensajes TIME_EXCEEDED, o a que se toman caminos alternativos  que terminan en ese mismo host para ambos TTLs. Dado que los RTT son muy similares para ambos hops, creemos que la primera explicación es mucho más probable.

http://www.net.in.tum.de/fileadmin/TUM/NET/NET-2012-08-1/
NET-2012-08-1\_02.pdf. 



Existen saltos 


\section{Conclusiones}

	Existen muchos factores que influyen en los tiempos de respuesta (el tipo de conexion a internet, la velocidad de internet, el ruteo de cada paquete, la congestion, los routers MPLS)
	Incluso usando una implementación de traceroute que mantenaga un único camino, problemas como congestión pueden el resultado (si solo se miran los tiempos)
	Mirar tiempo + IP del hop da una mejor idea (por ejemplo, se puede desetimar asaltos dentro de una misma red)

	Queda probar considerar los saltos entre subredes, en lugar de entre hops.
	El servicio de geolocalizacion puede ser erroneo.
	

Más tamaño de ráfaga tambien es mas riesgoso, porque hay mas chances de que varios paquetes sigan caminos distintos, haya mas varianza y el prefiltrado no funcione.

\section{Introducción}

Internet es posible gracias a un conjunto de miles de redes interconectadas entre sí.
La conexion entre redes de distintos continentes se realiza por medio de cables submarinos capaces de transportar grandes volumenes de datos por segundo.
En este trabajo experimentaremos con herramientas y técnicas frecuentemente utilizadas para el análisis de redes.
Analizaremos que tan factible resulta utilizar los datos estadisticos conseguidos mediante una implementacion propia de un traceroute, para detectar saltos intercontinentales en las rutas que atraviesan los paquetes en Internet.  
Se nos pide utilizar como objetivos cuatro universidades ubicadas en diferentes partes del mundo, y a distancias variadas.

%%Los conceptos teóricos sobre los que basaremos el analisis se presentan a continuación: 
\section{Marco teórico}

\subsection{Protocolo de Mensajes de Control de Internet- ICMP}

Forma parte del conjunto de protocolos IP (RFC 792). Los mensajes ICMP son comúnmente generados en respuesta a errores en los datagramas de IP para diagnóstico y ruteo. Estos mensajes son construidos en el nivel de capa de red y se encuentran dentro de los paquetes de IP estándar.

\subsection{Traceroute}

Es una herramienta de diagnóstico que permite seguir la pista de los paquetes que vienen de un host. Se obtiene además una estadistica del RTT o latencia de red de eso paquetes.
Exiten varias maneras de implementar esta herramienta, usualmente se envian paquetes IP donde se incrementa progresivamente el campo Time To Live (TTL), este campo sirve para que un paquete no permanezca en la red de forma indefinida. Cuando un nodo de la red elimina un paquete, envía al emisor un mensaje ICMP indicando una incidencia. Traceroute usa esta respuesta para averiguar la dirección IP del nodo que desechó el paquete, y asi sucesivamente hasta que el paquete llegue a su destino.

\subsection {Round Trip Time}

El RTT es el tiempo que tarda un paquete en ir y volver desde un nodo A (el origen) a un nodo B (el destino) dentro de una red. 
Cuando se trata de enlaces punto a punto, se define como 2 * Delay. 
Si bien a nivel de enlace puede realizarse una estimación relativamente confiable del Delay a partir de variables conocidas (ancho de banda, velocidad de propagación del medio, etc), a nivel de red el RTT de un paquete IP queda sujeto a la ruta que este toma.
Así, el RTT de un paquete que viaja entre varias redes interconectadas depende de variables desconocidadas de los enlace intermedios, y empiezan a cobrar mayor importancia factores como la congestión de los routers intermedios. Por lo tanto, en este trabajo solo consideraremos al RTT en su sentido empírico
	

\section{Resultados}
En esta sección mostraremos los resultados de los experimentos.

\subsection{Experimento 1:}

Las entropías calculadas para las 2 fuentes de información propuestas fueron:

\begin{center}
\begin{tabular}{ l r }
   Fuente& Entropía  \\
\hline
   1 & 1.371641 \\
   2 & 0.570425 \\
 \end{tabular}
\captionof{table}{entropias}
\label{entropiashogar}
\end{center}

Probabilidades para el modelo 1 según IPs:
%\begin{center}
%\begin{tabular}{ l c r }
%   Fuente& IP &probabilidad \\
%\hline
%   1 & 10.0.0.3  &0.864341085271 \\
%   2 & 10.0.0.2& 0.135658914729\\
%\caption{modelo 1}
% \end{tabular}
%\captionof{table}{modelo 1}
%\end{center}

\begin{center}
\begin{tabular}{ l r }
   IP &Probabilidad \\
\hline
192.168.1.138	& 0.3125 \\
192.168.1.103	& 0.0625 \\
192.168.1.137	& 0.1875 \\
192.168.1.1		& 0.3125 \\
192.168.1.143	& 0.125 \\
 \end{tabular}
\captionof{table}{Modelo 1}
\label{probabilidadesModel1}
\end{center}

Probabilidades para el modelo 2 según IPs:
\begin{center}
\begin{tabular}{ l r }
   IP &Probabilidad \\
\hline
192.168.1.1		& 0.9017 \\
192.168.1.103	& 0.0173 \\
192.168.1.143 	& 0.0115 \\
192.168.1.137	& 0.0173 \\
192.168.1.136	& 0.0057 \\
192.168.1.138	& 0.0462 \\
 \end{tabular}
\captionof{table}{Modelo 2}
\label{probabilidadesModel2}
\end{center}

\begin{figure}[H]
	\centering
	\includegraphics[scale = 0.8]{graficos/emiso_is_at_hogar.pdf}
	\caption{Histograma Prob./IPs Modelo 1}
	\label{histogramaprobabilidadesModel1}
\end{figure}


\begin{figure}[H]
	\centering
	\includegraphics[scale = 0.8]{graficos/recep_who_has_hogar.pdf}
	\caption{Histograma Prob./IPs Modelo 2}
	\label{histogramaprobabilidadesModel2}
\end{figure}

\newpage
\subsection{Experimento 2:}

\begin{center}
\begin{tabular}{ l r }
   Fuente& Entropía  \\
\hline
   1 & 3.533684 \\
   2 & 4.352357 \\
 \end{tabular}
\captionof{table}{entropias experimento 2}
\label{entropiasexperimento2}
 %\end{tabular}
\end{center}

En el caso del primer modelo había muchas IP's con la misma probabilidad y datos no significativos que tuvimos que eliminar,
debido a que perdía mucha claridad el gráfico. Pusimos las IP's con probabilidades más significativas.

\begin{figure}[H]
	\centering
	\includegraphics[scale = 0.6]{graficos/emiso_is_at_empresa.pdf}
	\caption{Histograma Prob./IPs Modelo 1}
	\label{histogramaprobabilidadesModel1}
\end{figure}

Con una \textbf{entropía} de \textbf{3.533684}.


En nuestro segundo experimento realizado en una red empresarial (clase B), una breve recolección de datos para el Modelo 2 (receptores de paquetes who-has) arrojó la siguiente información:

\begin{figure}[H]
	\centering
	\includegraphics[scale = 0.6]{graficos/recep_who_has_empresa.pdf}
	\caption{Histograma Prob./IPs Modelo 1}
	\label{histogramaprobabilidadesatos}
\end{figure}

Con una \textbf{entropía} de \textbf{4.352357}.

\section{Envio de paquetes}
En esta sección vamos a ver graficamente la cantidad de paquetes que se envian entre si los nodos de la red,con la esperanza de poder 
sacar algunas conclusiones. Los pesos en las aristas indican la cantidad de paquetes enviados de una IP a otra.

\subsection{Experimento 1}


\begin{figure}[H]
	\centering
	\includegraphics[scale = 0.6]{graficos/emisoras_is_at_red_hogar.png}
	\caption{paquetes is at emitidos en una red hogareña}
      \label{emisorasisat}
\end{figure}




\begin{figure}[H]
	\centering
	\includegraphics[scale = 0.5]{graficos/receptoras_who_has_red_hogar.png}
	\caption{paquetes who has en una red hogareña}
	\label{emisoraswhohas}	
\end{figure}

\newpage


\subsection{Experimento 2}
En este caso como se trataba de una red empresarial tuvimos que tomar recortes del grafo de envío de paquetes, ya que de lo contrario
se tornaba ilegible. Capturamos solo partes que colaboran con el objetivo de recolectar nodos distinguidos. 
Aclaración: En los 2 siguientes gráficos, el nodo apuntado representa a la IP origen mientras que el nodo apuntador es la IP destino.


\begin{figure}[H]
	\centering
	\includegraphics[scale = 0.5]{graficos/receptores_who_has_parte1_atos.png}
	\caption{paquetes who has en una empresarial}
	\label{receptoraswhohasatos}	
\end{figure}


\begin{figure}[H]
	\centering
	\includegraphics[scale = 0.5]{graficos/receptores_who_has_atos_parte2.png}
	\caption{paquetes who has en una red empresarial(segundo nodo)}
	\label{receptoraswhohasatos2 }	
\end{figure}

Aclaración: En los 2 siguientes gráficos, el nodo apuntado representa a la IP destino mientras que el nodo apuntador es la IP origen.

\begin{figure}[H]
	\centering
	\includegraphics[scale = 0.5]{graficos/emisoras_is_at_red_atos.png}
	\caption{paquetes is at en una empresarial}
	\label{emisorasisatatos1 }	
\end{figure}


\begin{figure}[H]
	\centering
	\includegraphics[scale = 0.3]{graficos/emisoras_is_at_atos.png}
	\caption{paquetes is at en una empresarial(segundo nodo)}
	\label{emisorasisatatos2 }
\end{figure}



